%%
%% HSM: June 2019
%%

\documentclass[english]{article}

%% \usepackage[latin9]{inputenc}
%% \usepackage{babel}
%%
%% other packages
\usepackage[all,cmtip]{xy}
\usepackage{url}
\usepackage{verbatim}
\usepackage{xspace}
\usepackage{amsmath}
\usepackage{amsthm}
\usepackage{thmtools}
\usepackage{amssymb}
\usepackage{color}
\usepackage{graphicx}
\usepackage{mathtools}
\usepackage[square,sort&compress,numbers]{natbib}
\usepackage[top=1in, bottom=1in, left=1in, right=1in]{geometry}
\usepackage[colorinlistoftodos]{todonotes}
\usepackage{hyperref}
\newcommand{\TODO}[1]{\todo[inline,color=red!10,size=\small]{#1}}
\renewcommand\thmcontinues[1]{Continued}
%%
%% ------------------------------------------------------------
\graphicspath{{./}{./Figures/}}
%% ------------------------------------------------------------
%%
\newtheorem{theorem}{Theorem}[section]
\newtheorem{corollary}[theorem]{Corollary}
\newtheorem{lemma}[theorem]{Lemma}
\newtheorem{proposition}[theorem]{Proposition}
%%\theoremstyle{definition}
\newtheorem{definition}[theorem]{Definition}
\theoremstyle{remark}
\newtheorem{remark}[theorem]{Remark}
\newtheorem{example}[theorem]{Example}
%%
\numberwithin{equation}{section}
%%
%% Layout:
\parindent0pt
\parskip6pt
%%
\renewcommand\labelitemi{\rule[0.12em]{0.4em}{0.4em}}
\renewcommand\labelitemii{\normalfont\bfseries \rule[0.15em]{0.3em}{0.3em}}

\def\EpiHiper{EpiHiper\xspace}
\def\EpiHiperVersion{1.0\xspace}
\def\rh#1{\hfill#1\hfill}
\def\thed#1{\textbf{#1}}
\def\hsp{\phantom{$\int\limits_a^b$}}
\def\SciDuct{SciDuct\xspace}

%%
%% Macros:

% Figure size.
\newcommand{\figsize}{0.27}
\newcommand{\expect}{\mathbb{E}}
\newcommand\numberthis{\addtocounter{equation}{1}\tag{\theequation}}
\def\th{\ensuremath{{}^\protect\text{\scriptsize th}}\xspace}

%% ----------------------------------------------------------------------
\begin{document}
%% ----------------------------------------------------------------------
\begin{titlepage}
\raggedleft\raggedbottom
\rule{1pt}{\textheight} % Vertical line
\hspace{0.05\textwidth} % Whitespace between the vertical line and title page text
%%
\parbox[b]{0.75\textwidth}{ % Paragraph box for holding the title page text, adjust the width to move the title page left or right on the page
  {\huge\bfseries NFA Threaded Router:\\[1ex]
    Design, Documentation, \\[1ex] and Scaling Studies}\\[9\baselineskip]
%%
{{\Large\textsc{Author(s):}\\[1ex]
%%
\phantom{\quad}Ryan Jung\\[1ex]
%%
\phantom{\quad}Henning S. Mortveit\\[1ex]
%%
}}\\[4\baselineskip]
%%
{{\Large\textsc{NSSAC Technical Report: No. 2019-TBD}}}\\[2\baselineskip]
%%
{{\Large\textsc{Status: \emph{Approved for NSSAC Internal Release}}}}\\[2\baselineskip]
%%
{{\Large\textsc{Contact:}\\[1ex]\phantom{\quad}Henning S. Mortveit (Henning.Mortveit@virginia.edu)}}\\[2\baselineskip]
%%
{{\Large\textsc{GitHub URL:}\\[1ex]\phantom{\quad}https://github.com/NSSAC/RE\_Router}}\\[7\baselineskip]
%%
\vspace*{\fill}
\vfill
%%
Network Systems Science and Advanced Computing\\
Biocomplexity Institute and Initiative\\
University of Virginia
}
\end{titlepage}
%% ----------------------------------------------------------------------


%%
\title{NFA Threaded Router: \\Design, Documentation, and Scaling Studies}
%%
\author{
Ryan Jung${}^a$ and
%%
Henning S. Mortveit${}^{a,b,\dag}$\footnote{$^\dag$
  Corresponding author: Henning S. Mortveit;\hfil\break
  email: henning.mortveit@virginia.edu;\hfil\break
  github URL: https://github.com/NSSAC/RE\_Router\hfil\break
  NSSAC Technical Report: No. TR 2019--TBD
}\\
%%
\\[1ex]\\
${}^a$Network Systems Science and Advanced Computing;\\
%%
${}^b$Department of Engineering Science and Environment;\\
%%
%%${}^c$Department of Computer Science;\\
%%
%%
University of Virginia
}
\maketitle


%%\keywords{Keywords: To add}
%%
\begin{abstract}
This technical report covers the threaded implementatin of the regular
expression (or NFA) constrained router of Jakob et al.

To be completed.

\end{abstract}
%%

%% ----------------------------------------------------------------------
\section*{Introduction}
\label{sec:introduction}

Describe the algorithm of Jakob et al~\cite{Barrett:00}.

\textbf{Report organization.} Section~\ref{sec:design} gives the
... Section~\ref{sec:scalingstudies} presents ... We conclude in
Section~\ref{sec:documentation} providing complete user instructions
as well as deploy constructions.

%% ----------------------------------------------------------------------
\section{Design}
\label{sec:design}

Give a careful diagram of the software design. You can treat the
router as a black box, no need to go into details for this part.


%% ----------------------------------------------------------------------
\section{Scaling Studies}
\label{sec:scalingstudies}

Outline of work:
\begin{itemize}
\item Construct e.g. a Python tool that takes as argument the node
  file of a network, and an integer $N$, and that produces a complete
  trip request file containing $N$ random trips. They will all have
  travel mode $0$ which correponds to automobile. It will likely be
  useful to have trip request files with 100, 1,000, 10,000, 100,000,
  and 1,000,000 requests. For testing, use the smaller request files;
  for the serious scaling studies, use the larger one.
\item Create a timing framework that can time and report the execution
  of the router. It may be helpful to separately time initialization
  code such as construction of networks.
\item Create a timing diagram giving time needed for computation as a
  function of the number of requested compute threads. Conclusions?
  Linear scaling? If not, why not? What happens if you request more
  threads than there are available cores? For each timing run, there
  may be fluctations due to other computations running on your
  computer/Rivanna. It will likely be useful to at least conduct two
  runs per setting.
\item If time permits, conduct a scaling study on Rivanna using
  multiple nodes.

\end{itemize}


%% ----------------------------------------------------------------------
\bibliographystyle{amsplain}
\bibliography{references}

%% ----------------------------------------------------------------------

\cleardoublepage
\appendix

%% ----------------------------------------------------------------------
\section{RE\_Router User Documentation}
\label{sec:documentation}

\TODO{HSM: may include deployment; running on a desktop; running on
  Rivanna using Slurm}

\TODO{HSM: Fix the following}

\begin{verbatim}
./EpiHiper -seed <integerNumericValue> -dbconn <jsonString> -config <configFilename>

  -seed: an optional non-negative integer to be used as seed for
    random number generation. It will be used by process 0 which in turn
    will generate and communicate random seed values to all remaining processes.

  -config: a required string specifying the EpiHiper configuration
   file. The configuration file must conform to the JSON schema for the
   EpiHiper input configuration.

  -dbconn: a string argument giving the JSON structure with information required to
    establish the database connection to the person trait database. This argument is
    required if any configuration file references a person trait database.
\end{verbatim}


%% ----------------------------------------------------------------------


\subsection{Input Files}
\label{sec:inputfiles}

The router uses the following input files:
\begin{itemize}
\item Network node file
\item Network link file
\item Etc ... to be completed.
\end{itemize}




\end{document}
